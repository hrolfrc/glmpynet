\documentclass[11pt, a4paper]{article}

% --- PACKAGES ---
\usepackage[utf8]{inputenc}
\usepackage[T1]{fontenc}
\usepackage{geometry}
\usepackage{enumitem} % For custom lists
\usepackage{hyperref} % For clickable links

% --- DOCUMENT SETUP ---
\geometry{a4paper, margin=1in}
\hypersetup{
    colorlinks=true,
    linkcolor=blue,
    urlcolor=cyan,
}

% --- DOCUMENT START ---
\begin{document}

\section*{Contributing to bib-ami}

First off, thank you for considering contributing to \texttt{bib-ami}! We welcome any help, from reporting a bug to suggesting a new feature. This project is maintained by volunteers in their spare time, so your contributions are highly valued.

To ensure a smooth and effective process for everyone, please read through these guidelines before you start.

\subsection*{How Can I Contribute?}

There are many ways to contribute, and all of them are appreciated.
We've ordered them from the most common and helpful ways to get involved.

\subsubsection*{1. Reporting Bugs}

This is one of the most valuable ways to contribute.
If you find a bug, please open an issue on our GitHub repository.
A great bug report is one that can be reproduced. Please include the following:

\begin{itemize}[leftmargin=*]
    \item \textbf{A clear and descriptive title:} e.g., "Fails to merge files with non-ASCII characters in path."
    \item \textbf{A description of the steps to reproduce the bug:} What command did you run? What was in your input files?
    \item \textbf{What you expected to happen.}
    \item \textbf{What actually happened:} Include any error messages or tracebacks.
    \item \textbf{Your environment:} Your operating system, Python version, and \texttt{bib-ami} version.
\end{itemize}

\subsubsection*{2. Suggesting Enhancements or New Features}

If you have an idea for a new feature or an improvement to an existing one, we'd love to hear it.
Please open an issue to start a discussion.

\begin{itemize}[leftmargin=*]
    \item \textbf{Start a discussion first:} Before you spend time writing code, please open an issue to propose your idea.
    This allows us to discuss the feature and make sure it aligns with the project's goals before any work is done.
    This is the best way to respect everyone's time.
    \item \textbf{Be clear and concise:} Explain the problem you're trying to solve and how your proposed feature would solve it.
\end{itemize}

\subsubsection*{3. Improving Documentation}

Clear documentation is essential.
If you find a typo, think a section is unclear, or believe something is missing, please don't hesitate to open an issue or submit a pull request with your suggested improvements. This is a fantastic way to make your first contribution.

\subsubsection*{4. Submitting Pull Requests}

We welcome pull requests for bug fixes and approved features.

\begin{itemize}[leftmargin=*]
    \item \textbf{Prerequisite:} Please ensure there is an open issue discussing the bug or feature before you submit a pull request. This helps us track the work and ensures your contribution is aligned with the project's direction.
    \item \textbf{Follow the process:}
        \begin{enumerate}
            \item Fork the repository and create your branch from \texttt{main}.
            \item Make your changes. Please adhere to the existing code style.
            \item Add tests for your changes. We value code quality and need to ensure the tool remains stable.
            \item Ensure the test suite passes.
            \item Submit your pull request, linking it to the relevant issue.
        \end{enumerate}
\end{itemize}

\subsection*{A Note on Time}

Please remember that this project is maintained on a volunteer basis.
While we will do our best to review issues and pull requests in a timely manner, there may be delays.
Starting with a discussion in an issue is the best way to ensure your efforts are not wasted.

Thank you again for your interest in making \texttt{bib-ami} better!

\end{document}
